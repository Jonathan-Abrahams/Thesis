\documentclass{article}
\usepackage[utf8]{inputenc}
\setlength{\parskip}{1em}
\usepackage{subcaption}


\title{Chapter 3}
\author{Jonathan S. Abrahams }
\date{October 2019}

\usepackage{natbib}
\usepackage{graphicx}

\begin{document}

\maketitle

\section{Results}

We found many CNVs but it was not clear what their impact is like.

GWAS are an attractive analyses to find genotype-phenotype links.

Large phenotypic datasets do not yet exist for BP. In anticipation of these,however, it is important to explore the data analysis of this kind of data. This may be able to inform the process of data generation to produce the clearest signal.

I therefore adapted existing methodologies in order to use GWAS to find genotype-phenotype links for both SNPs and CNVs.

This would be particularly good for BP where little is known about many genes.

One barrier to GWAS,however, is the highly clonal structure of BP.

I investigated if key mutations would be separable from the phylogenetic structure of BP.

What is highly interesting about BP,however, is that alleles that appear to be under positive selection do not sweep through the population after a single introduction but instead occur multiple times independently. This means that despite the clonal nature of BP, it may be a highly interesting species to study using GWAS.
\subsubsection{Considerations}

GWAS can be undertaken on kmers or a list of defined mutations. K-mers can provide a much more powerful and flexible solution than analysing a list of mutations generated by other utilities. due to the unique nature of K-mer, their  analysis  is fast and using certain algoirthms,   both gene presence/absence in addition to nucleotide changes can be described. They were unsuitable for this analysis,however, and this is outlined below.


%for two inter-related reasons: Tools are not optimised to use K-mer abunduence beyond presence and absence and the computational load of counting K-mers in raw reads.

%Assemblies with resolved dupes are lacking, its a great idea to try and use predicted CNVs
The majority of bacterial GWA analyses rely on assembled genomes rather than raw reads due to their reduced redundancy and complexity. However as was outlined in Chapter 2, because of the long and complex nature of CNVs in BP there is a lack of CNV-resolved assemblies available. There is at most 6 isolates with CNVs at the same locus in the manualyl resolved genomes dataset in Chapter 2. This means that using only CNV-resolved assemblies would likely produce an underpowered analysis in addition to a very limited scope of the analysis with only 1 out of the 11 hotspots being possible to analyse. There is,however, many more CNVs predicted but which remain unassembled. The ability to use these CNVs in a GWA analysis would vastly improve the power of the analysis and increase the scope of the research.

%Predicted CNVs can be used in a k-mer analysis on raw data.
%This is complex because Current methods treat k-mers as presence or absence.
%

As the CNV predictions in Chapter 2 were based on the read depth of short read data, the same signal (increased coverage of CNV loci) can be exploited again to produce K-mer abundunces instead of read-depth. This strategy,however, is highly complex. Firstly, current methodologies are not tuned to analyse K-mer abundances beyond their presence or absence. Therefore this means that some isolates having twice as many K-mers is an invisible signal to the analysis and gets analysed merely as K-mer presence rather than a possible K-mer duplication. It is likely possible,however, that current methods can be tweaked to incorporate such signals and was briefly investigated here. 

Analysing K-mers in raw sequence data was was not pursued more broadly as a strategy, however, as the second reason that this analysis is complicated is that it computationally intensive. Due to the increased quantity of DNA sequences, this makes the K-mer dataset much larger which in trial tests I found required in excess of 60gb of RAM to analyse, which was the limit of the server used. This was therefore beyond the scope of the current research.

%As read-depth was used to predict CNVs in Chapter 2, it is possible to use this same signal from the data in a K-mer based analysis. It is possible to analyse K-mer abundunces from raw sequence data.



%It could be demonstrated that existing K-mer counting methods could be manipulated to provide a genotype to a GWA algorithm. However,  the lack of resolved CNVs in pertussis means that K-mer counting using assemblies will likely not produce statistically significant results. 
 
 %In Chapter 2 it could be demonstrated that there were a maximum of 6 isolates with the same CNV. Whilst it is possible this would be sufficient to find a genotype-phenotype link, due to the difficulties in generating further duplication-resolved genomes, this method was not generic and repeatable for other loci.



%It is possible to use unassembled sequence data to count K-mers and generate a signal of CNVs for a sample. 

I therefore pursued using predefined mutations as the `genotype' in the GWAS. This allowed a relatively light computational load whilst still giving the flexibility of using any mutation type-a deletion,SNP or duplication.

\subsubsection{Detecting linkage in key SNPs}


I wanted to investigate a variety of different SNPs which are theorised to be important to virulence and the resurgence of BP.
%Need to think of a better reason to slim the dataset down

We used a condensed dataset composed of the 720 isolates with at least 1 copy number variable gene compared to B1917 in order to ease the computational load and simplify the analysis.

Due to no detectable inter-chromosomal recombination occurring, BP has very high linkage disequilibrium. However, many of the key mutations are also known to be homoplasies. I therefore hypothesised that many of the homoplasy mutations would give highly statistically significant results but that mutations that have high allele frequency due to clonal expansion may be 

Running Pyseer with a `perfect phenotype' effectively is establishing linkage between SNPs but accounting for population structure. E.g are there multiple homoplasies.


\subsubsection{Erythromycin resistance}

%Great study that in the discussion links to other studies where resistance was found

%https://wwwnc.cdc.gov/eid/article/25/12/18-1836_article#r11

This is a mutation that has spread rapidly in China, it is likely to be under heavy selection there. It has also been reported globally.



This SNP occured in 3 independent lineages in China. 

Antibiotic resistance is colloquially known as the `low hanging fruit' of GWAS as there is often a very simple and direct link between a single SNP and resistance.

W


\subsubsection{PTXp3}

PTXp3 is known to be a homoplasy mutation. It is therefore a strong candidate for GWAS analysis. However it is also known that this SNP is not solely responsible for enhanced pertussis toxin production. It is therefore likely that some mutations that are linked to the ptxP3 SNP also contribute to the phenotype. However in this analysis, as the phenotype is not naturally made , the results simply indicate there are multiple homoplasies linked that are not explained by the population structure.

I therefore ran Pyseer with

\subsubsection{Deletions}

\subsubsection{Duplications}


At this moment in time, it 
\subsubsection{}
\end{document}
