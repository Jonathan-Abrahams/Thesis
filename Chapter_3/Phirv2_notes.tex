\documentclass{article}
\usepackage[utf8]{inputenc}
\setlength{\parskip}{1em}
\usepackage{subcaption}


\title{PhiRV2 notes}
\author{Jonathan S. Abrahams }
\date{October 2019}

\usepackage{natbib}
\usepackage{graphicx}

\begin{document}

\maketitle

\section{What is phirv2}

phirv2 is a phage that is integrated into the MTB genome. It is absent in BCG.
\section{phirv2 distribution}

All TB strains prior to 2014 contained the phirv2 prophage. The deletion of phirv2 was only observed once, in the 2016 study (https://www.tandfonline.com/doi/full/10.1080/07391102.2015.1022602)

\section{Known role of phirv2 genes}

phirv2 genes were upregulated when hypoxia was induced using chemials in TB. (fan 2016). Part of the `starvation stimulon'. https://www.sciencedirect.com/science/article/pii/S168411821400036X


phirv2 genes were found to be differentially translated in strains that were treated with rifampicin. The same study then went on to predict an interaction with many antibiotics via rpoB. (https://link.springer.com/article/10.1007%2Fs00203-017-1448-0 )


https://www.sciencedirect.com/science/article/pii/S1286457916301460?via%3Dihub

BCG vaccine strain lacks this region. Therefore immune response to latent TB in vaccinated individuals is not optimal. INcluding these antigens in the vaccine appears to improve its efficacy.
https://www.sciencedirect.com/science/article/pii/S1286457916301460#bib149.

Isoniazid treatment led to an increase in latent TB's rv2659c response to ifn-y.

ΦRv2 integrates into a trna gene.

Therefore phirv2 deletion may reduce the ability of TB to become latent and was found to be associated iwth antibiotic sensitivity.

Why would this be the case?

Can we confirm it is likly to have multiple independent loss events?
\section{link of phirv2 to antibiotic resistance}

\section{RDrio}

RDrio is a Region of Difference (RD) that has been noticed predominantely in Brazil and South America. It has, however, also been noticed in other strains from around the world. In Brazil, RDrio strains make up 26-40 percent of the population
\subsection{RDrio link to resistance}


\end{document}
