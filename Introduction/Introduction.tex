\documentclass{article}
\usepackage[utf8]{inputenc}
\setlength{\parskip}{1em}
\usepackage{subcaption}


\title{Chapter 1}
\author{Jonathan S. Abrahams }
\date{October 2019}

\usepackage{natbib}
\usepackage{graphicx}

\begin{document}

\maketitle



\section{Mistakes are the key to success}
%Genomes are prone to errors due to machinery inconsistencies
The observed diversity of living organisms can be simplistically attributed to a series of mistakes in DNA.  The maintenance, copying and sexual proccesses that DNA undergoes are facilitated by the cellular machinery which, whilst a rare occurrence, are capable of making `mistakes' (now refered to as mutations). As genotypes ultimately interact with the enviroment through a phenotype (the functional impact of a genotype), genetic diversity generates phenotypic diversity in a a population leading to a range of individuals of different suitability to that enviroment.

As all individuals (including genetic elements) strive to ultimately maximise their own success, there is competition between individuals over limited resources.  As competition drives innovation, the fittest (in relation to the specific enviroment at that time) individuals survive- this is known as natural selection. There are two major components to natural selection: Stochastic processes of DNA mutation and  selection acting on phenotypes.

%Not all errors are the same




%Many of these processes are fairly ubiquitous, but the errors happen at certain rates, with some being more common than others 
\subsection{How mutations occur}
%Not all mutations are the same
A range of mutations occur in DNA  by various different mechanisms. Whilst the focus of this thesis is on structural variations, a brief overview of the mechanisms of other types of mutations are covered in order to compare and contrast mutation types and serve as prelude.

%Rate
Due to the different processes behind each mutation type they arise at different rates.


\subsubsection{nucleotide changes}
%What are they
The most well studied mutation type is a single base change ( a single nucleotide polymorphism or SNP). A SNP will occur during DNA replication when the DNA polymerase recruits the wrong nucleotide to the newly synthesised DNA strand combined with when the exonuclease of the polymerase complex fails to rectify the wrong base being incorperated.
 
 
 %https://www.ncbi.nlm.nih.gov/books/NBK459274/
 %
%How do they occur

%How can this be harnessed
\subsubsection{Gene gain and loss}

\section{The unique biology of structural variations}
%Fantastic paper on dupes. The intro paragraphs are amazing 
%https://www.ncbi.nlm.nih.gov/pmc/articles/PMC4315931/
%https://www.ncbi.nlm.nih.gov/pmc/articles/PMC4142968/
\subsection{How SVs are formed}
%Sv occur by DSB repair using homologous recombinionat
DNA is constantly `assualted' by both endogeonous (for example: DNA replication or by metabolic byproducts) or exogenous (for example oxidative stress,radiation or chemical sources). This damage can cause the formation of single stranded break in the DNA helix or, more rarely, a double stranded break which is composed of two single stranded breaks at a complementary base pair. These breaks cannot re-anneal back together correctly due to chemical modification of their structure and therefore the correct repair of this error requires assistance from a set of enzymes and pathways. 


Whilst this is a universal feature of DNA, the mechanisms by which different bacterial species repair such breaks is not understood nor which pathways are preffered in what species. Therefore it is nececary to draw conclusions form model organisms and assume they are broadly representative, for example I will here use E.coli as an example of a gram negative species. 

Whilst a diverse range of pathways can be used to repair different kinds of breaks, 
%Need to  check this statement and its importance
%If these genome aberrations are not repaired, however, they can lead to replication fork stalls.
\subsection{Genome instability}



%really good one . links to many other reviews
%https://onlinelibrary-wiley-com.ezproxy1.bath.ac.uk/doi/epdf/10.1111/j.1574-6976.2011.00272.x
%NJEH https://www.sciencedirect.com/science/article/pii/S0960982209012457
\subsection{Selection favours `fit' mutations}

\subsection{Strutural variations as an overlooked class of mutation in bacteria}

Changes to the DNA base sequence are the most obvious and well studied mutation type in biology whereas structural variations, due to their nature, are harder to study leading to fewer mentions in the literature. 



\subsubsection{SVs in eukaryotic studies}
Despite some of the fundamental discoveries about SVs being made in bacteria, most studies of SVs are in eukaryotic organisms (in particular deletions and duplications) in eukaryotes. In humans SVs are known to predispose people to XYZ and to directly cause QPR. Furthermore, SVs can be easier to study in eukaroytes (in particular humans).
%Generation time means that SV are more stable in humans-from one day to the next they are the same, generally.

%Trio studies can be used- The children should (by and large) just have CNVs that originated from the parents. Use the CNVnator data.

%Also involved in mitochondria
\subsubsection{SVs differ in bacteria}


%In contrast to eukaryotes, bacteria have super fast generation times. 
\subsection{}


\section{BP biology}
Bordetella pertussis is a Gram-negative bacterium which is the main causative agent of the human respiratory disease whooping cough. B. pertussis has speciated from a B. bronchiseptica-like ancestor to become a host restricted pathogen (1,2). This process has occurred primarily via genome reduction: the B. bronchiseptica genome is around 5.4Mbp whereas the B. pertussis genome is around 4.1Mbp, involving loss of over 1000 genes during speciation, and has been driven primarily by deletions arising from recombination between Insertion Sequence (IS) elements (2,3). Genomes of B. pertussis strains include over 240 copies of IS481, with far fewer copies of IS1663 and IS1002. Gene erosion in B. pertussis appears to be on-going and sporadic IS-mediated deletions and disruptions provide subtle differences in gene content between strains (4–6), but there is little understanding of the effects. 

\subsection{Bordetella pertussis evolution}

\subsection{Epidemiology}

\subsubsection{Pre-NGS epi}

\subsubsection{Post-NGS epi}


\subsection{}


\section{Long read sequencing}
\subsection{Nanopore and pacbio}
\subsection{How long reads help with SV}
\subsection{Hurdles yet to be overcome with long reads}



Using the most popular metric of genetic diversity, single nucleotide polymorphisms (SNPs), B. pertussis is a species with extraordinarily low diversity leading to its description as a monomorph (7,8). More detailed analyses of B. pertussis genome sequences have been limited by the inability to generate closed genome assemblies from short-read sequencing data, as the reads do not span IS481 (1043 bp), and the assembly produces many contigs- consistently in excess of the number of IS481 copies. Recent advances in long-read sequencing, notably by Pacific Biosciences (PacBio) and Oxford Nanopore Technologies (Nanopore), has enabled routine generation of closed genome assemblies for B. pertussis (8–12). Subsequent comparative analyses have revealed that intragenomic recombination between IS481 causes genomic rearrangement and that a large number of different genome orders exist among circulating B. pertussis isolates (8). The effect of rearrangement on B. pertussis phenotype remains unknown but moving genes between leading and lagging strands and to different locations in the chromosome would be expected to alter their expression (13,14). Likewise, transcription from IS element promoters can affect neighbouring genes, and different copies of IS481 exhibit different transcriptional activities (15). Rearrangements that shuffle IS element-neighbouring gene combinations might, therefore, elicit changes in gene expression profiles both locally and genome-wide.

In addition to deletion and rearrangement, IS-mediated recombination can result in duplication. Twelve copy number variants (CNVs) in B. pertussis have been described and studies with sufficient genomic data have resolved them as tandem repeats (5,9,10,16–18). Duplication of a region containing cyaA (encoding adenylate cyclase-haemolysin) increased haemolytic activity and it was noted that this duplication was highly unstable (17). These serendipitous observations suggest that CNVs are a poorly characterised contributor to genetic diversity among B. pertussis. However, to date there has been no systematic analysis of CNVs in B. pertussis and indeed systematic analysis of structural variants at the species level is rare for bacteria, although it is relatively common in eukaryotic organisms. In this study we sought to catalogue CNVs in B. pertussis, utilising publicly available genomic data, which is overwhelmingly derived from short-read sequencing platforms. 

Among genomic data from 2430 B. pertussis isolates we found 191 which contained evidence of CNVs and identified that 94 \% of CNVs occur at 11 ‘hotspot’ loci. Some CNVs were very large, exceeding 300 kb in length. We reveal that some regions are present in multi-copy, and thus use the term copy number variant (CNV) rather than duplication. We contextualise this information using phylogenetics and find that strains containing similar CNVs are often distantly related, suggesting that CNVs at hotspot loci arise independently. Also, we confirm that laboratory grown populations of cells contain a mixture of copy numbers suggesting that CNV formation is a dynamic process, at least at some loci. Our study revealed novel genetic variation among B. pertussis isolates and provides a blueprint for investigation of CNVs in other bacteria, particularly those with high numbers of repeats.


\end{document}